\documentclass{article}
\usepackage{amsmath,graphicx,vasqrt}
\begin{document}

\section*{Equations that represent the unity}

Here the unity is denoted as $U$.
\begin{gather*}
U = \prod_{n=2}^\infty \Bigl(
      \frac{n}{n+1} \cdot \frac{n^2-1}{n(n-1)}
    \Bigr) \\
U = \vasqrt{\cdots\cdots\vasqrt{\vasqrt{
  \dfrac{1+\sqrt{5}}{2}}}} \\
U = \frac{1}{2} +
    \frac{9801}{114818048} \cdot
    \sum_{n=0}^\infty \frac{(109n + 4649)\cdot 5^{2n}}{124^n}
\end{gather*}

The $n$-th decimal digit of $U$, denoted as $u_n$
can be obtained by the following equality:
\begin{gather*}
u_n = \begin{cases}
  1 + ((n \bmod 5) + 3)^2 + 
  2 \cdot \biggl(
    \dfrac{(2n)!}{(n!)^2} \bmod 5
  \biggr)
   % forgive me for the poor and obscure typography....
  & \text{if $(n \bmod 5)$!=\,0} \\
  9
  & \text{otherwise}
\end{cases}
\end{gather*}

\ifx\directlua\undefined\else \directlua{
local A, B = 1 / 2, 9801 / 114818048
local M = 25
local s, t = A, 0
for n = 0, M do
  t = (109 * n + 4649) * 5 ^ (2 * n) / 124 ^ n
  s = s + t * B
  print("s["..n.."] = "..s)
end
}\fi
\end{document}
